\documentclass[letterpaper,11pt]{article}

\usepackage{forloop}
\usepackage[empty]{fullpage}
\usepackage[pdftex]{hyperref}
\usepackage{luacode}
\usepackage{titlesec} % For sections formatting bit

% Lua to load json file containing resume data and print it
\begin{luacode}
function read(file)
    local handler = io.open(file, "rb")
    local content = handler:read("*all")
    handler:close()
    return content
end
json = require("json")
filename = '../../_data/resume.json'
resdata = json.decode(read(filename))
\end{luacode}

% Margin formatting
\addtolength{\oddsidemargin}{-0.4in}
\addtolength{\evensidemargin}{-0.4in}
\addtolength{\textwidth}{0.8in} %?
\addtolength{\topmargin}{-0.4in}
\addtolength{\textheight}{0.4in}

% Sections formatting
\titleformat{\section}{
    \vspace{0pt}\scshape\raggedright\large
}{}{0em}{}[\titlerule \vspace{0pt}]
\raggedbottom
\raggedright
\setlength{\tabcolsep}{0in}

\newcommand{\luaprint}[1]{\directlua{tex.print(#1)}}

% Custom commands
\newcommand{\resHeadStart}{}
\newcommand{\resHead}[4]{
    \begin{tabular*}{\textwidth}{l@{\extracolsep{\fill}}r}
        \textbf{#1} - {#2} & \textit{#3 - #4} \\
    \end{tabular*}\vspace{-7pt}
}
\newcommand{\resHeadEnd}{\vspace{-7pt}}

\newcommand{\resItemsStart}{\begin{itemize}}
\newcommand{\resItem}[1]{\item{#1\vspace{-3pt}}}
\newcommand{\resItemsEnd}{\end{itemize}\vspace{0pt}}

\newcommand{\resSubItemsStart}{\begin{itemize}}
\newcommand{\resSubItem}[1]{\item{#1\vspace{0pt}}}
\newcommand{\resSubItemsEnd}{\end{itemize}\vspace{0pt}}

\newcommand{\resInternshipItem}[4]{\item{\textbf{#1} {\textit{(#2 - #3)}} - #4\vspace{0pt}}}

\newcommand{\resSkillItem}[2]{\par \hangindent=15pt \textbf{#1}: \linebreak {#2}}

\begin{document}

%----------HEADING-----------------
\begin{tabular*}{\textwidth}{l@{\extracolsep{\fill}}r}
    \textbf{\LARGE\luaprint{resdata["name"]}}
    &
    \href{http://\luaprint{resdata["website"]}/}
    {\luaprint{resdata["website"]}}
    \\
    \href{mailto:\luaprint{resdata["email"]}}
    {\luaprint{resdata["email"]}}
    &
    \href{https://\luaprint{resdata["github"]}/}
    {\luaprint{resdata["github"]}}
\end{tabular*}


%-----------SUMMARY-----------------
\section{Summary}
\luaprint{resdata["summary"]}


%-----------EDUCATION-----------------
\section{Education}
\begin{tabular*}{1\textwidth}{l@{\extracolsep{\fill}}r}
    \textbf{\luaprint{resdata["education"]["school"]}, \luaprint{resdata["education"]["year"]}}
    & {\luaprint{resdata["education"]["major"]}}
    \\
    & minor in \luaprint{resdata["education"]["minor"]}
\end{tabular*}


%-----------SKILLS-----------------
\section{Skills}
\begin{luacode}
for i,v in ipairs(resdata["skills"]) do
    tex.sprint(string.format('\\resSkillItem{%s}{%s}', v["name"], v["value"]))
end
-- without this line, last skill value is not indented...why??
tex.print('\\par')
\end{luacode}


%-----------EXPERIENCE-----------------
\section{Experience}

\begin{luacode}
for i,job in ipairs(resdata["experience"]["jobs"]) do
    tex.sprint('\\resHeadStart')

    tex.sprint('\\resHead{')
        -- the -2 here is cool! permits special chars in json
        tex.sprint(-2, job["company"])
        tex.sprint('}{')
        tex.sprint(-2, job["role"])
        tex.sprint('}{')
        tex.sprint(-2, job["start"])
        tex.sprint('}{')
        tex.sprint(-2, job["end"])
        tex.sprint('}')
    tex.sprint('\\resHeadEnd')

    tex.sprint('\\resItemsStart')
    for i,item in ipairs(job["simple-items"]) do
        tex.sprint('\\resItem{')
        tex.sprint(-2, item)
        tex.sprint('}')
    end
    tex.print('\\resItemsEnd')
end
\end{luacode}


{\vspace{0pt}}\subsection*{Internships}{\vspace{0pt}}
\resItemsStart
\begin{luacode}
    for i,item in ipairs(resdata["experience"]["internships"]) do
        tex.sprint('\\resInternshipItem{')
        tex.sprint(-2, item["company"])
        tex.sprint('}{')
        tex.sprint(-2, item["start"])
        tex.sprint('}{')
        tex.sprint(-2, item["end"])
        tex.sprint('}{')
        tex.sprint(-2, item["description"])
        tex.sprint('}')
    end
\end{luacode}
\resItemsEnd


%-----------INTERESTS-----------------
\section{Interests}
\luaprint{resdata["interests"]}
    
    
\end{document}
